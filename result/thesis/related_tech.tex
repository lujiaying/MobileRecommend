\chapter{推荐系统概述以及大数据相关技术}

\section{推荐系统概述}

  \subsection{基于规则的推荐}

  \subsection{协同过滤}

  \subsection{点击率预估+排序}

  \subsection{组合模型}

\section{大数据技术概述}

  \subsection{Hadoop}
  Hadoop(http://hadoop.apache.org/)是一个由Apache基金会所开发的分布式系统基础架构。Hadoop是根据Google公司发表的MapReduce和Google文件系统的论文自行实现而成。
  Hadoop框架透明地为应用提供可靠性和数据移动。它实现了名为MapReduce的编程范式:应用程序被分区成许多小部分,而每个部分都能在集群中的任意节点上运行或重新运行。此外,Hadoop还提供了分布式文件系统\gls*{HDFS},用以存储所有计算节点的数据,这为整个集群带来了非常高的带宽。MapReduce和分布式文件系统的设计,使得整个框架能够自动处理节点故障。它使应用程序与成千上万的独立计算的电脑和PB级的数据。现在普遍认为整个Apache Hadoop“平台”包括Hadoop内核、MapReduce、Hadoop分布式文件系统(HDFS)以及一些相关项目,有Apache Hive和Apache HBase等等。
  用户可以在不了解分布式底层细节的情况下,开发分布式程序。充分利用集群的威力进行高速运算和存储。HDFS有高容错性的特点,并且设计用来部署在低廉的(low-cost)硬件上;而且它提供高吞吐量(high throughput)来访问应用程序的数据,适合那些有着超大数据集(large data set)的应用程序。HDFS放宽了POSIX的要求,可以以流的形式访问(streaming access)文件系统中的数据。Hadoop的框架最核心的设计就是:HDFS和MapReduce。HDFS为海量的数据提供了存储框架,则MapReduce为海量的数据提供了计算框架。

    \subsubsection{Hadoop的优点}
    Hadoop是一个能够对大量数据进行分布式处理的软件框架。 Hadoop 以一种可靠、高效、可伸缩的方式进行数据处理。
    Hadoop 是可靠的,因为它假设计算元素和存储会失败,因此它维护多个工作数据副本,确保能够针对失败的节点重新分布处理。Hadoop 是高效的,因为它以并行的方式工作,通过并行处理加快处理速度。Hadoop 还是可伸缩的,能够处理 PB 级数据。此外,Hadoop 依赖于社区服务,因此它的成本比较低,任何人都可以使用。
    Hadoop是一个能够让用户轻松架构和使用的分布式计算平台。用户可以轻松地在Hadoop上开发和运行处理海量数据的应用程序。它主要有以下几个优点:
    \begin{enumerate}
    \item 高可靠性。Hadoop按位存储和处理数据的能力值得人们信赖。
    \item 高扩展性。Hadoop是在可用的计算机集簇间分配数据并完成计算任务的,这些集簇可以方便地扩展到数以千计的节点中。
    \item 高效性。Hadoop能够在节点之间动态地移动数据,并保证各个节点的动态平衡,因此处理速度非常快。
    \item 高容错性。Hadoop能够自动保存数据的多个副本,并且能够自动将失败的任务重新分配。
    \item 低成本。与一体机、商用数据仓库以及QlikView、Yonghong Z-Suite等数据集市相比,hadoop是开源的,项目的软件成本因此会大大降低。
    \end{enumerate}
    Hadoop带有用Java语言编写的框架,因此运行在 Linux 生产平台上是非常理想的。Hadoop同样提供其他语言的接口,包括Scala、C++、Shell、Python、Ruby等,其他语言通过Streaming API与Hadoop交互。
    \subsubsection{hadoop大数据处理的意义}
    Hadoop得以在大数据处理应用中广泛应用得益于其自身在数据提取、变形和加载(ETL)方面上的天然优势。Hadoop的分布式架构,将大数据处理引擎尽可能的靠近存储,对例如像ETL这样的批处理操作相对合适,因为类似这样操作的批处理结果可以直接走向存储。Hadoop的MapReduce功能实现了将单个任务打碎,并将碎片任务(Map)发送到多个节点上,之后再以单个数据集的形式加载(Reduce)到数据仓库里。

  \subsection{Spark}

%% 本章参考文献
\ifx\usechapbib\empty
\nocite{BSTcontrol}
\setcounter{NAT@ctr}{0}
\bibliographystyle{buptgraduatethesis}
\bibliography{bare_thesis}
\fi

