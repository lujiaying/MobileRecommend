%%
%% This is file `example/bare_thesis.tex',
%% generated with the docstrip utility.
%%
%% The original source files were:
%%
%% install/buptgraduatethesis.dtx  (with options: `bare-thesis')
%% 
%% This file is a part of the example of BUPTGraduateThesis.
%% 

\documentclass[%
  degree=master,%
  classlevel=open,%
  mathfont=cm,%
  dedication=false,%
  chapbib=true,%
  finish=print,%
  driver=xetex]{buptgraduatethesis}

%% 自定义导言区
%% 在这里添加你需要的宏包、自定义命令、环境等
%% \usepackage{...}
%% \DeclareMathOperator{\CT}{H}
%% \DeclareMathOperator{\Cov}{Cov}
\def\BUPTThesis{\textsc{BUPT}\-\textsc{Thesis}}

%% 在这里添加图片文件搜索目录
\graphicspath{{../}}
%% 自定义导言区结束

%% 加载缩略语定义
%%
%% This is file `example/metadata.tex',
%% generated with the docstrip utility.
%%
%% The original source files were:
%%
%% install/buptgraduatethesis.dtx  (with options: `metadata')
%% 
%% This file is a part of the example of BUPTGraduateThesis.
%% 

%% 涉密论文保密年限
\classdur{三年}

%% 学号
\studentid{2014140099}

%% 论文题目
\ctitle{基于电商大数据的推荐算法研究}
%%\etitle{Example of BUPT Graduate Thesis \LaTeXe{} Template}

%% 申请学位
\cdegree{工程硕士}

%% 院系名称
\cdepartment{信息与通信工程学院}

%% 专业名称
\cmajor{电子与通信工程}

%% 你的姓名
\cauthor{卢嘉颖}

%% 你导师的姓名
\csupervisor{李勇}

%% 日期自动生成,也可以取消注释下面一行,自行指定日期
%%\cdate{\CJKdigits{2013}年\CJKnumber{12}月\CJKnumber{25}日}

%% 中文摘要
\cabstract{%
  本课题主要针对电子商务产生的大数据进行研究,采用分布式系统存储并处理这些数据量大、多变、速度快且价值密度低的数据,比较协同过滤、逻辑回归、随即森林、GBDT等多种机器学习推荐算法的准确率和召回率,并应用于分布式系统上,分析各算法的优劣,最终提出有创新性和适应于海量数据的算法。
}

%% 中文关键词,关键词之间用 \kwsep 分割
\ckeywords{推荐系统 \kwsep 机器学习 \kwsep 分布式系统 \kwsep spark}

%% 英文摘要
%%\eabstract{%
%%  The Chinese and English abstract should appear after the declaration page.
%%  The abstract should present the core of the research work, especially the purpose and importance of the research, the method adopted, the results, and the conclusion.
%%
%%  Key words are terms selected for documentation indexing, which should present the main contributions of the thesis.
%%  Key words are aligned at the bottom left side of the abstract content.
%%  Key words should be seperated by spaces but not any other symbols.
%%}

%% 英文关键词,也用 \kwsep 分割
%%\ekeywords{%
%%  \TeX \kwsep \LaTeX \kwsep xeCJK \kwsep template \kwsep typesetting \kwsep thesis}


\loadglsentries{acronyms}

%% 攻读学位期间发表论文
%% 用 \newcite{<suffix>}{<caption>} 声明不同的论文类型(例如: 期刊论文、会议论文等)。每一个类型的对应的 .bib 文件用 \bibliography<suffix> 命令加载,用 \nocite<suffix> 命令引用。具体请参考 pubs.tex 中的示例
\newcite{jrnl}{期刊论文}
\newcite{conf}{会议论文}
\newcite{patent}{专利}

\begin{document}
%% 声明前置部分
\makefrontmatter

%% 生成主要符号对照表
%%
%% This is file `example/notations.tex',
%% generated with the docstrip utility.
%%
%% The original source files were:
%%
%% install/buptgraduatethesis.dtx  (with options: `notations')
%% 
%% This file is a part of the example of BUPTGraduateThesis.
%% 

\begin{listofnotations}
\item [$(\cdot)^*$] 复共轭
\item [$(\cdot)^{\mathrm T}$] 矩阵转置
\item [$(\cdot)^{\mathrm H}$] 矩阵共轭转置
\item [$\mathbf{X}$] 矩阵或向量
\item [$\mathcal{A}$] 集合
\item [$\mathcal{A}\times\mathcal{B}$]
  集合 $\mathcal{A}$ 与集合 $\mathcal{B}$ 的 Cartesian 积,即 $\mathcal{A}\times\mathcal{B}=\{(a,b):a\in\mathcal{A},b\in\mathcal{B}\}$
\end{listofnotations}


%% 主体部分
\mainmatter
%% 用\include{}命令引用各章.tex文件
%%
%% This is file `example/ch_intro.tex',
%% generated with the docstrip utility.
%%
%% The original source files were:
%%
%% install/buptgraduatethesis.dtx  (with options: `ch-intro')
%% 
%% This file is a part of the example of BUPTGraduateThesis.
%% 

\chapter{绪论}
随着移动互联网、物联网、云计算等新兴信息技术在社会各个领域的广泛应用,全球数据量正呈现出前所未有的指数型增长态势。与此同时,数据类型的丰富性及来源的多样性、数据产生的高速性与分析的实时性、数据的低价值密度等复杂特征日益凸显,标志着“大数据”时代的到来。大数据同云计算、物联网一样,是信息技术领域的重大技术变革。

大数据的产生在很大程度上降低了消费者和企业之间的信息不对称程度。一方面,企业通过多元化的信息获取渠道掌握消费者的全面信息,提供的产品和服务更具针对性;另一方面,分散孤立的消费者同样通过多种渠道了解产品的各种信息,需求逐步呈现出个性化和多样化趋势。交易双方信息的愈加透明促进消费者与生产企业之间更加互动,消费者的个性化需求成为生产企业关注的核心。因此,大数据等新一代信息技术的发展使得消费者的地位日益重要,推动电子商务的价值创造方式发生转变,生产企业以消费者为中心创造高度差异化的产品和服务,并且引导消费者参与产品生产和价值创造。
通过对海量和复杂的数据进行收集、整理与分析,不仅能够提升对社会经济发展的预测能力,而且能够不断地在各领域创新商业模式。本课题将分析大数据背景下的电子商务推荐算法的创新及其在大数据应用面临的挑战。

互联网的出现和普及给用户带来了大量的信息,满足了用户在信息时代对信息的需求,但随着网络的迅速发展而带来的网上信息量的大幅增长,使得用户在面对大量信息时无法从中获得对自己真正有用的那部分信息,对信息的使用效率反而降低了,这就是所谓的信息超载(informationoverload)问题。
解决信息超载问题一个非常有潜力的办法是推荐系统,它是根据用户的信息需求、兴趣等,将用户感兴趣的信息、产品等推荐给用户的个性化信息推荐系统。和搜索引擎相比推荐系统通过研究用户的兴趣偏好,进行个性化计算,由系统发现用户的兴趣点,从而引导用户发现自己的信息需求。一个好的推荐系统不仅能为用户提供个性化的服务,还能和用户之间建立密切关系,让用户对推荐产生依赖。
推荐系统现已广泛应用于很多领域,其中最典型并具有良好的发展和应用前景的领域就是电子商务领域。同时学术界对推荐系统的研究热度一直很高,逐步形成了一门独立的学科。


\section{大数据背景下推荐算法的研究现状和发展趋势}
当前,大数据已深耕于经济领域并创造了巨大的经济价值,美国将大数据上升为国家战略,英国开展了“数据权”运动,欧盟提出了开放数据战略,而中国也发布了大数据标准化白皮书。可以说,世界各主要经济体都将大数据视作未来国家竞争力的重要组成部分。

在电子商务领域,大数据技术的发展给很多企业带来了广阔的发展机会。传统电子商务创新主要局限在电子商务的效率、便利化、营销方式等方面,大数据技术的广泛应用给电子商务的模式创新带来机遇。基于大数据的电子商务创新主要在于提炼大数据的价值并将其应用于电子商务的各个流程,形成新的商业模式\cite{大数据背景下电子商务的价值创造与模式创新}。其中,由推荐算法延伸出的推荐系统满足了大数据时代的消费者的个性需求,使得按需定制变为可能,得以创造实时化、差异化的产品及服务以满足各种长尾群体的需求。

自从1992年施乐的科学家为了解决信息负载的问题,第一次提出协同过滤算法,个性化推荐已经经过了二十几年的发展。1998年,林登和他的同事申请了“item-to-item”协同过滤技术的专利,经过多年的实践,亚马逊宣称销售的推荐占比可以占到整个销售GMV(Gross Merchandise Volume,即年度成交总额)的30\%以上。随后Netflix举办的推荐算法优化竞赛,吸引了数万个团队参与角逐,期间有上百种的算法进行融合尝试,加快了推荐系统的发展,其中SVD(Sigular Value Decomposition,即奇异值分解,一种正交矩阵分解法)和Gavin Potter跨界的引入心理学的方法进行建模,在诸多算法中脱颖而出。其中,矩阵分解的核心是将一个非常稀疏的用户评分矩阵R分解为两个矩阵:User特性的矩阵P和Item特性的矩阵Q,用P和Q相乘的结果R'来拟合原来的评分矩阵R,使得矩阵R'在R的非零元素那些位置上的值尽量接近R中的元素,通过定义R和R'之间的距离,把矩阵分解转化成梯度下降等求解的局部最优解问题。
与此同时,Pandora、LinkedIn、Hulu、Last.fm等一些网站在个性化推荐领域都展开了不同程度的尝试,使得推荐系统在垂直领域有了不少突破性进展,但是在全品类的电商、综合的广告营销上,进展还是缓慢,仍然有很多的工作需要探索。特别是在全品类的电商中,单个模型在母婴品类的效果还比较好,但在其他品类就可能很差,很多时候需要根据品类、推荐栏位、场景等不同,设计不同的模型。同时由于用户、SKU不停地增加,需要定期对数据进行重新分析,对模型进行更新,但是定期对模型进行更新,无法保证推荐的实时性,一段时间后,由于模型训练也要相当时间,可能传统的批处理的Hadoop的方法,无法再缩短更新频率,最终推荐效果会因为实时性问题达到一个瓶颈。


\section{论文的主要内容}
本论文介绍了经典的推荐算法,并结合大数据时代的特点给出算法的并行化版本。此外,使用Boosting,Bagging等思想进行组合推荐,试图通过组合来优化或弥补各推荐算法的弱点。

全文内容安排如下:
\begin{enumerate}
\item 第一章:绪论。提出了推荐系统的研究意义和国内外研究现状,并简要介绍了推荐系统面临的问题,随后阐述了XXX对推荐系统的意义。最后提出了本文的焦点研究和研究成果。
\item 第二章:推荐系统概述以及大数据相关技术。简要介绍了推荐系统的定义、解决的问题,以及常见的推荐算法。结合时下最先进的大数据技术,提出推荐系统与大数据技术的结合方案。
\item 第三章:电商网站中的商品推荐问题。
\item 第四章:商品推荐问题的解决方案,详细叙述几种方案的具体步骤、比较各方案的性能及结果分析。
\item 第五章:使用Hadoop、Spark构建电商推荐系统。
\item 第六章:结论与工作展望。
\end{enumerate}

%% 本章参考文献
\ifx\usechapbib\empty
\nocite{BSTcontrol}
\setcounter{NAT@ctr}{0}
\bibliographystyle{buptgraduatethesis}
\bibliography{bare_thesis}
\fi

%%
%% This is file `example/ch_concln.tex',
%% generated with the docstrip utility.
%%
%% The original source files were:
%%
%% install/buptgraduatethesis.dtx  (with options: `ch-concln')
%% 
%% This file is a part of the example of BUPTGraduateThesis.
%% 

\chapter{功能测试}
脚注使用带圈数字的表示方法,此处为示例 1\footnote{测试脚注一} 和示例 2\footnote{测试脚注二}。

缩略语的功能非常强大,例如首次出现 \gls*{WTT} 和非首次出现 \gls*{WTT} 时将显示不同的内容。

参考文献可以使用\cite{BUPT_Thesis_Format_2014}和\onlinecite{BUPT_Thesis_Format_2004}的表示方法。

\section{三国演义}
《三国演义》\cite{SANGUOYANYI}是中国第一部长篇章回体历史演义的小说,以描写战争为主,反映了蜀(汉)、魏、吴三个政治集团之间的政治和军事斗争,大致分为黄巾之乱、董卓之乱、群雄逐鹿、三国鼎立、三国归晋五大部分。

在广阔的背景下,上演了一幕幕波澜起伏、气势磅礴的战争场面,成功刻画了近五百个人物形象,其中曹操、刘备、孙权、诸葛亮、周瑜、关羽、张飞等人物形象脍炙人口,其中诸葛亮是作者心目中的“贤相”的化身,他具有“鞠躬尽瘁,死而后已”的高风亮节,具有近世济民再造太平盛世的雄心壮志,而且作者还赋予他呼风唤雨、神机妙算的奇异本领。
曹操是一位奸雄,他生活的信条是“宁教我负天下人,休教天下人负我”,既有雄才大略,又残暴奸诈,是一个政治野心家阴谋家这与历史上的真曹操是不可混同的。
关羽“威猛刚毅”、“义重如山”。
但他的义气是以个人恩怨为前提的,并非国家民族之大义。
刘备被作者塑造成为仁民爱物、视贤下士、知人善任的仁君典型。

\subsection{长坂坡}
京剧《长坂坡》\cite{CHANGBANPO}是依据《三国演义》改编的京剧传统剧目。

故事叙述:刘备自烧屯新野之后,弃樊城,阻襄阳,一路率引军民,流离败走,穷促万分。
关羽、诸葛亮,已先后遣往夏口,乞救于刘琦未返,刘备等往投江陵暂驻,中途经过当阳,驻扎景山之下。
忽然曹操大兵,漫山遍野追至,夤夜厮杀,刘备众大败,及天明检点随从只余百余骑,刘备家眷及赵云、简雍、二糜等将,均不知下落,其余百姓,亦均散失殆尽。
此时赵云因于阿斗及甘、糜二夫人等失散,遂单骑冲突,四处找寻主眷,沓无下落。
往回三数次,遇见简雍被创卧地,始略知失踪处所。
赵云先救出简雍,令回,再往军中及百姓中搜访,先救甘夫人于难民队,同时又救糜竺,亲自护送至长坂坡,令糜竺保甘嫂先行,折身再回,觅糜嫂及阿斗。
途中刺落夏侯恩,收获青釭宝剑,七次冲入重围,方得百姓指引,得见糜夫人抱阿斗坐于坍墙枯井之旁啼哭。
夫人身受数创,不能行走。
赵云叩见,极力请夫人上马,欲保护而出。
夫人深知大义,惟以阿斗为托,己则以愿死报主,免累赵云,赵云再三安慰催行,力任无妨,夫人再三不可,亦促赵云速行。
继见赵云坚待不去,恐且迟延遇寇,乃跳身入井,以速赵云之行。
赵云大惊,尚踌躇设法营救,则曹军人马已至,不得已推墙掩井,解甲藏阿斗于胸前,忽忽上马,厮杀夺围欲出。
此时曹操大兵云集,群矢于赵云一身,赵云在核心,东斩西杀,虽不败辱,而屡濒于厄。
幸曹操爱勇将,赖徐庶乘间说曹操,以生擒勿伤,传令全军,始得完肤而返。

测试所有参考文献类型\cite{CITATION_BOOK,CITATION_ARTICLE,CITATION_PROCEEDINGS,CITATION_INPROCEEDINGS,CITATION_TECHREPORT,CITATION_STANDARD,CITATION_PATENT,CITATION_NEWSPAPER,CITATION_ELECTRONIC}。

%% 本章参考文献
\ifx\usechapbib\empty
\nocite{BSTcontrol}
\setcounter{NAT@ctr}{0}
\bibliographystyle{buptgraduatethesis}
\bibliography{bare_thesis}
\fi


%% 附录部分

%% 如果有两个或两个以上的附录, 使用appendix环境
\begin{appendix}
  %%
%% This is file `example/app_lhospital.tex',
%% generated with the docstrip utility.
%%
%% The original source files were:
%%
%% install/buptgraduatethesis.dtx  (with options: `app-lhospital')
%% 
%% This file is a part of the example of BUPTGraduateThesis.
%% 

\chapter{不定型($0/0$)极限的计算}
\begin{theorem}[L'Hospital法则]
  若
  \begin{enumerate}
  \item 当 $x \to a$ 时,函数 $f(x)$ 和 $g(x)$ 都趋于零;
  \item 在点 $a$ 某去心邻域内,$f'(x)$ 和 $g'(x)$ 都存在,且 $g'(x)\neq 0$;
  \item $\displaystyle\lim_{x \to a} \dfrac{f'(x)}{g'(x)}$ 存在(或为无穷大),
  \end{enumerate}
  那么
  \begin{align}
    \label{eq:app:lhospital}
    \lim_{x \to a} \frac{f(x)}{g(x)} = \lim_{x \to a} \frac{f'(x)}{g'(x)}.
  \end{align}
\end{theorem}
\begin{proof}
  以下只证明两函数 $f(x)$ 和 $g(x)$ 在 $x = a$ 为光滑函数的情形。
  由于 $f(a) = g(a) = 0$,原极限可以重写为
  \begin{align*}
    \lim_{x \to a} \frac{f(x) - f(a)}{g(x) - g(a)}.
  \end{align*}
  对分子分母同时除以 $(x - a)$,得到
  \begin{align*}
    \lim_{x \to a} \frac{%
      \dfrac{f(x) - f(a)}{x - a}
    }{%
      \dfrac{g(x) - g(a)}{x - a}
    } &
    = \frac{%
      \displaystyle\lim_{x \to a} \frac{f(x) - f(a)}{x - a}
    }{%
      \displaystyle\lim_{x \to a} \frac{g(x) - g(a)}{x - a}
    }.
  \end{align*}
  分子分母各得一差商极限,即函数 $f(x)$ 和 $g(x)$ 分别在 $x = a$ 处的导数
  \begin{align*}
    \lim_{x \to a} \frac{f(x)}{g(x)} &
    = \frac{f'(a)}{g'(a)}.
  \end{align*}
  由光滑函数的导函数必为一光滑函数,故 \eqref{eq:app:lhospital} 得证。
\end{proof}

  %%
%% This is file `example/app_lhospital.tex',
%% generated with the docstrip utility.
%%
%% The original source files were:
%%
%% install/buptgraduatethesis.dtx  (with options: `app-lhospital')
%% 
%% This file is a part of the example of BUPTGraduateThesis.
%% 
\chapter{优化}

\section{梯度下降法}

\section{拉格朗日乘子法}

  % 自动抽取生成缩略语表作为附录A
  \tableofacronyms
  % 用\input{}添加其他的附录
  % \input{...}
\end{appendix}

%% 如果只有一个附录, 使用appendix*环境
%% \begin{appendix*}
%%   % 自动抽取生成缩略语表作为附录A
%%   % \tableofacronyms
%% \end{appendix*}

\ifx\usechapbib\undefined
\bibliographystyle{buptgraduatethesis}
\bibliography{bare_thesis}
\fi

\backmatter
%% 致谢
\ifx\ispeerreview\undefined
%%
%% This is file `example/ackgmt.tex',
%% generated with the docstrip utility.
%%
%% The original source files were:
%%
%% install/buptgraduatethesis.dtx  (with options: `ackgmt')
%% 
%% This file is a part of the example of BUPTGraduateThesis.
%% 

\begin{acknowledgement}
  %% 感谢所有你应该感谢的人
  感谢Donald Ervin Knuth.
\end{acknowledgement}

\fi

%% 在读期间论文发表情况
%%
%% This is file `example/pubs.tex',
%% generated with the docstrip utility.
%%
%% The original source files were:
%%
%% install/buptgraduatethesis.dtx  (with options: `pubs')
%% 
%% This file is a part of the example of BUPTGraduateThesis.
%% 

%% 发表论文列表

%% 攻读学位期间发表论文列表用 tableofpublications 环境产生。需要
%% 在 bare_thesis.tex 的导言区用 \newcite{<name>}{<caption>} 声明不同类
%% 型的论文,具体见导言区说明。
%% 根据各类论文发表数量设置\setbiblabelwidth{<num>},用于控制发表论文序号的对齐位置。
%% 例如:发表conf类论文数量为个位数,则<num>=1;发表jrnl类论文数量为两位数,则<num>=10;
\begin{tableofpublications}
  \thispagestyle{bupt@pubheadings}%
  \setcounter{NAT@ctr}{0}
  \setbiblabelwidth{1}
  \bibliographyjrnl{pubs}
  \nocitejrnl{paper1}

  \setbiblabelwidth{1}
  \bibliographyconf{pubs}
  \nociteconf{paper2}

  \setbiblabelwidth{1}
  \bibliographypatent{pubs}
  \nocitepatent{patent1}
\end{tableofpublications}


\newpage
\end{document}
